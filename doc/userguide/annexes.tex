%
%                               POK header
%
% The following file is a part of the POK project. Any modification should
% be made according to the POK licence. You CANNOT use this file or a part
% of a file for your own project.
%
% For more information on the POK licence, please see our LICENCE FILE
%
% Please follow the coding guidelines described in doc/CODING_GUIDELINES
%
%                                      Copyright (c) 2007-2025 POK team

\chapter{Annexes}

   \section{Terms}
   \begin{itemize}
      \item[$\bullet$] \textbf{AADL}: AADL stands for Architecture Analysis and
         Design Language. It provides modeling facilities to represent a system
         with their properties and requirements.

      \item[$\bullet$] \textbf{Leon3}: A processor architecture developped by
         the European Space Agency.

      \item[$\bullet$] \textbf{Ocarina}:
         AADL compiler developed by TELECOM ParisTech. It is used by the POK
         project to automatically generate configuration, deployment and
         application code.

      \item[$\bullet$] \textbf{PowerPC}:
         Architecture popular in the embedded domain.

      \item[$\bullet$] \textbf{QEMU}:
         A general-purpose emulator that runs on various platforms and emulates
         different processors (such as \intel x86 or \powerpc).
   \end{itemize}


   \section{Resources}
   \label{annex-url}
   \begin{itemize}
      \item[$\bullet$] POK website: \textit{http://pok.gunnm.org}
      \item[$\bullet$] Ocarina website: \textit{http://aadl.telecom-paristech.fr}
      \item[$\bullet$] QEMU website: \textit{http://www.qemu.com}
      \item[$\bullet$] Cheddar:
      \textit{http://beru.univ-brest.fr/~singhoff/cheddar/}
      \item[$\bullet$] MacPorts: \textit{http://www.macports.org}
   \end{itemize}

   \section{POK property set for the \aadl}
   \lstinputlisting{pok_properties.aadl}

   \section{AADL library}
   This is not C code but an \aadl library that can be used with your own
   models. When you use this library, you don't have to specify all your
   components and properties, just use predefined components to generate your
   application.

   \lstinputlisting{aadl-library.aadl}

   \section{ARINC653 property set for the \aadl}
   \lstinputlisting{arinc653_properties.aadl}

   \section{Network example, modeling of device drivers}
   \label{annex-device-driver}
   \lstinputlisting{sample-network-rtl8029.aadl}
   \lstinputlisting{sample-network-runtime.aadl}
   \lstinputlisting{sample-network-main.aadl}

