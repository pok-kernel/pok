%
%                               POK header
%
% The following file is a part of the POK project. Any modification should
% be made according to the POK licence. You CANNOT use this file or a part
% of a file for your own project.
%
% For more information on the POK licence, please see our LICENCE FILE
%
% Please follow the coding guidelines described in doc/CODING_GUIDELINES
%
%                                      Copyright (c) 2007-2025 POK team

\chapter{Examples}

\section{Assurance Quality}
At each source code change, the developper must compile and check that examples
compile fine on all supported architectures.

Consequently, the available examples with the release compiles. Sometimes, you
can experience some errors since the examples are not run. If you think you find
a bug, please report it to the developper team.

\section{List of provided examples}
This section details each example and the services they use.
These examples are available in each release of POK.

\begin{itemize}
   \item
      \textbf{arinc653-blackboard} :
         test the blackboard service
         of the ARINC653 layer. This test relies on an AADL
         model that describe a blackboards between two tasks.
   
   \item
      \textbf{arinc653-buffer}:
         test the buffer service of ARINC653.
         It uses AADL models.

   \item
      \textbf{arinc653-queueing}:
         test ARINC653 queuing ports with AADL.

   \item
      \textbf{arinc653-sampling}:
         test ARINC653 sampling ports with AADL.

   \item
      \textbf{arinc653-threads}:
         test ARINC653 processes instanciation.
         Uses AADL models.

   \item
      \textbf{case-study-aerotech09}:
         An ARINC653 examples case study
         for the AEROTECH09 conference. It uses two partitions
         that communication temperature across inter-partitions
         communications.
   \item
      \textbf{case-study-sigada09}: a system that contain three partitions
      with different design patterns (non-preemptive scheduler,
      ravenscar partition, queued buffers). This example was used
      as use-case for a publication in the SIGAda conference (SIGAda09).
   \item
      \textbf{case-study-mils}: a distributed with two nodes that communicate
      data at different security levels. Data are encrypted using cipher
      algorithms provided by the libpok layer. 
   \item
      \textbf{case-study-ardupilot}: a case-study made from application code
      found in the ardupilot project. See.
      \texttt{http://code.google.com/p/ardupilot/} for documentation about this
      application code.
   \item
      \textbf{case-study-integrated}: a case-study that shows we can use POK for
      real avionics architecture. For that, we define an avionics architecture
      using AADL and generates code for POK using Ocarina. The initial model is
      defined by the Software Engineering Instute (SEI). See
      \texttt{http://www.aadl.info/aadl/currentsite/examplemodel.html\#Family}
      for more information about this initial model. Note that we convert this
      model from AADLv1 to AADLv2 to make it working with Ocarina/POK.
   \item
      \textbf{data-arrays}
         Test the use of array types and their use
         in communication with queuing ports. Use AADL
         models to describe types.

   \item
      \textbf{data-arrays2}
         Test the use of array types and their use
         with sampling ports. Use AADL model to describe types.

   \item
      \textbf{esterel}
         Use of a third-party language as application-layer.
         In this case, we use Esterel generated code. Use
         AADL models.

   \item
      \textbf{events}
         Test the use of events ports between threads located
         in the same partition. DO NOT use AADL models.

   \item
      \textbf{exceptions-handled}:
         Test the exceptions catching (recovery
         handlers). Use AADL models.

   \item
      \textbf{heterogeneous-partitions}:
         Define two partitions with
         different architectures. Demonstrate that the build
         system can generate, build and run different modules
         that have different architectures.


   \item
      \textbf{libmath}:
         Test the inclusion of the libmath library in the libpok layer.
         Use AADL model.

   \item
      \textbf{lustre-academic}:
         Test the inclusion of Lustre application code
         inside partition. Use AADL models.

   \item
      \textbf{middleware-blackboard}:
         Test the use of blackboard service. Use AADL models.

   \item
      \textbf{middleware-buffer}:
         Test the use of buffer service. Use AADL models.

   \item
      \textbf{middleware-buffer-timed}:
         Test the use of buffer service with timeout.
         Use AADL models.

   \item
      \textbf{middleware-queueing}:
         Test the use of queuing port service. Use AADL models.

   \item
      \textbf{middleware-queueing-timed}:
         Test the use of queuing port service with timeout.
         Use AADL models.

   \item
      \textbf{middleware-sampling}:
         Test the use of sampling port service.
         Use AADL models.

   \item
      \textbf{mutexes}:
         Test mutex service (POK layer).
         Use AADL models.

   \item
      \textbf{mutexes-timed}:
         Test mutex service with timeout. Use AADL models.
         
   \item
      \textbf{network}:
         Test network driver (rtl8029) on x86 architecture.
         Use AADL models.

   \item
      \textbf{partitions-scheduling}:
         Example with different schedulers.
         Use AADL models.

   \item
      \textbf{partitions-threads}:
         Test thread instanciation (POK layer). Use AADL models.

   \item
      \textbf{semaphores}:
         Test the use of semaphors (POK layer). Do not use AADL
         models.

   \item
      \textbf{simulink}: Test the inclusion of simulink code. Do not
         work since it needs a dedicated runtime to work. It needs additional
         services in the libpok layer to work. This additional work is not so
         difficult to provide, we just need time !
\end{itemize}



