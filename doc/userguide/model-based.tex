%
%                               POK header
%
% The following file is a part of the POK project. Any modification should
% be made according to the POK licence. You CANNOT use this file or a part
% of a file for your own project.
%
% For more information on the POK licence, please see our LICENCE FILE
%
% Please follow the coding guidelines described in doc/CODING_GUIDELINES
%
%                                      Copyright (c) 2007-2025 POK team

\chapter{Automatic configuration and configuration with AADL models}
   \label{chapter-automatic-configuration}

   \section{Proposed development process}
   Using \aadl models can help system designers and developers in the
   implementation of partitioned architectures. \pok can be configured
   automatically using \aadl models. In fact, the \aadl is very efficient for
   the design of real-time embedded systems: designers specify their
   architecture with respect to their specificites and requirements. Then, the
   \ocarina toolsuite analyzes the architecture and automatically generates
   code for \pok.

   The code generation process automatically configures the kernel and
   the partitions. The developpers should provide the application-level code.
   This application-level code can be traditional code (Ada, C) or 
   application models (Simulink, Scade, etc.).

   Our code generator was integrated in the \ocarina \aadl toolsuite. It is a
   popular toolsuite for \aadl models handling. It provides several
   functionnalities, such as models analysis, verification and code generation.
   In the context of \pok, we rely on these functionnalities to verify and
   automatically implement the system.

   \onehugefig{imgs/mde-process}{Model-Based development
   process}{figure:mde-process}

   The development process is illustrated in the figure
   \ref{figure:mde-process}: the developper provides AADL models, the code
   generator creates code that configures kernel and libpok layers. Compilation
   and integration is automatically achieved by the toolchain and creates final
   binary runnable on embedded hardware.


   \section{Use the pok toolchain for model analysis, validation, code generation, compilation and execution (the \texttt{pok-toolchain.pl} script)}
   We provide a toolchain that provides the following functionnalities:
   \begin{enumerate}
      \item
         \textbf{Model analysis}: check that your \aadl model is correct.
      \item
         \textbf{Model validation}: validate the requirements specified in the model
      \item
         \textbf{Code generation}: automatically generate the code for its execution with
         \pok
      \item
         \textbf{Compilation}: automatically compile and create binaries
   \end{enumerate}


      \subsection{Use the \texttt{pok-toolchain.pl} script}
      The toolchain is implemented in a script called \texttt{pok-toolchain.pl}.
      This script is used to perform the different actions of the development
      process. This script has the following options:
      \begin{itemize}
         \item
            \texttt{\-\-models=} is a \textbf{REQUIRED} option. It specifies the
            \aadl models you use for this system. For example, you can specify
            \texttt{\-\-models=model1.aadl,model2.aadl}. This is the list of
            your models.
         \item
            \texttt{\-\-no\-generate}: do not generate the code. By default, the
            toolchain generates the code from \aadl models.
         \item
            \texttt{\-\-no\-run}: do not run the generated systems. By default,
            the toolchain generates code and run generated systems.
         \item
            \texttt{\-\-no\-check}: do not validate the architecture.
         \item
            \texttt{\-\-root=system\_name}: Specify the root system of your
            architecture. If your models contain several \texttt{system}
            components, you need to specify what is the \aadl root
            \texttt{system} component.
         \item
            \texttt{\-\-arinc653}: use \arinc code generation patterns.
      \end{itemize}

      \subsection{Example of use}
      The following line will generate \arinc-compliant code from
      \texttt{model1.aadl} model.
      \begin{verbatim}
      pok-toolchain.pl --models=model1.aadl --arinc653
      \end{verbatim}

      The following line will generate code and compile it, but will not run generated system.
      \begin{verbatim}
      pok-toolchain.pl --models=model1.aadl --no-run
      \end{verbatim}

   \section{Model validation}
   Our toolchain automatically validates models requirements before generate
   code. It was made to help system designer in the verification of its
   architecture.

   Our validation process is based on \ocarina and the REAL language, which is a
   constraint language for the \aadl. Its quite similar than OCL language
   (designed for UML), except that is specific to \aadl and thus, makes easier
   the validation of \aadl model. You can have additional information about
   \ocarina and REAL on \textit{http://www.aadl.telecom-paristech.fr}. With
   REAL, the user defines one or several \textit{theorems} that express what we
   want to check.

   There is a list of the theorems used in the \pok toolchain and what we
   verify:
   \begin{enumerate}
      \item
         \textbf{MILS requirements enforcements}: we check that each partition
         has one security level and connected partitions share the same security
         levels. For that, the underlying runtime and the connections should
         support appropriate security levels.
      \item
         \textbf{Bell-Lapadula and Biba security policies}: for connected
         partitions, we check the Bell-Lapadula and Biba security policies (no
         read-up/write-down, \ldots). With that, we ensure that the architecture
         is compliant with strict security guidelines.
      \item
         \textbf{Memory requirements}: we check that required size by a
         partition is less important than the size of its bounded memory
         component. In other words, we check that the memory segment can store
         the content of the partition. We also check that the requirements
         described on partitions are correct regarding their content (threads,
         \texttt{subprograms} size, \ldots).
      \item
         \textbf{Scheduling requirements (Major Time Frame)}: for each
         \texttt{processor} component, we check that the major time frame is
         equal to the sum of partitions slots. We also check that each partition
         has at least one time frame to execute their threads.
      \item
         \textbf{Architecture correctness}: we check that models contain memory
         components with the appropriate properties. We also check that
         \texttt{process} components are bound to \texttt{virtual processor}
         components.
   \end{enumerate}

   \section{\pok properties for the \aadl}
   The \aadl can use user-defined property sets to add specific properties on
   \aadl components. On our side, we define our own \aadl properties, added to
   \aadl components to describe some specific behavior.

   The \pok property set for the \aadl can be found in the annex section.


   In addition, \pok and its associated \aadl toolsuite (\ocarina) supports the
   \arinc annex of the \aadl. So, you can use models that enforces the \arinc
   annex with \pok. The \arinc property set for the \aadl is included in the
   annex section of this document.

   \section{Modeling patterns}
   This section describes the code generation patterns used to generate code.
   So, it explain the mapping between \aadl models and generated code. To
   understand this section, you have to know the \aadl. You can find tutorials
   in the internet about this modeling language (Wikipedia can be a good
   starting point).

      \subsection{Kernel}
      The kernel is mapped with the \aadl \texttt{processor} component. If the
      architecture is a partitioned architecture, it contains partitions runtime
      (\aadl \texttt{virtual processor} components).

         \subsubsection{Scheduling}
         The scheduling requirements are specified in \texttt{process}
         components properties. The \texttt{POK::Slots} and
         \texttt{POK::Slots\_Allocation} properties indicate the different time
         slots for partitions execution (in case of a partitioned architecture).

         In addition, the \texttt{POK::Scheduler} is used to describe the
         scheduler of the processor. If we implement an \arinc architecture, the
         scheduler would be static.

      \subsection{Device drivers}
         In POK, device drivers are executed in partitions. It means that you
         must embedds your code in partitions and drivers are isolated in terms
         of time and space. Consequently, drivers rely on the kernel to gain
         access to hardware resources (I/O, DMA and so on).

         To do that, AADL components are considered as partitions. So, when your
         model contains an AADL device, the underlying code generator consider
         that it is a partition. So, you have to \textbf{associate} device
         components with \texttt{virtual processor} components to indicate the
         partition runtime of your driver.

         However, the device driver cannot describe the actual implementation of
         the driver. For that, we use the \texttt{Implemented\_As} property.
         This property points to an abstract component that contains the
         implementation of our driver. Annexes of the current document provide
         an example of the modeling of a driver (see section
         \label{annex-network-example}): the \texttt{driver\_rtl8029}
         \texttt{abstract} component models the driver by defining a process
         that contains threads. These threads handle the device and perform
         function calls to access to hardware resources.

         However, for each device, POK must know which device you are actually
         using. So, you have to specify the \texttt{POK::Device\_Name} property.
         It is just a string that indicate which device driver you are using
         with this device component.

         In addition, for network device that represent ethernet buses, you must
         specify the hardware address (also known as the MAC address). For that,
         we have the property \texttt{POK::Hw\_Addr}. This property must be
         associated with a device component.

         \paragraph{Supported device drivers\\}
         At this time, we only support one device driver : the realtek 8029
         ethernet controller. This device is simulated by QEMU and thus, can be
         easily tested and simulated on every computer that uses QEMU.

         However, implementing other device driver can be easily achieved, by
         changing the \texttt{Device\_Name} property in the model and adding
         some functions in the \texttt{libpok} layer of POK.

         \paragraph{Case study that defines a device driver\\}
         You can find an example of an implementation of a device driver in the
         \texttt{examples/network} directory of each POK release. It defines two
         ARINC653 module that communicate across an ethernet network. Each
         module contains one partition that communicate over the network. You
         can have more information by browsing the \texttt{examples/network}
         directory.


      \subsection{Partitions}
         In case of a partitioned architecture, we need to describe partitions
         in your \aadl model. In that case, partitions are mapped with two \aadl
         components: \texttt{process} and \texttt{virtual processor}.

         The \texttt{virtual processor} models the runtime of the partition (its
         scheduler, needed functionalities and so on).


         We associate the \texttt{virtual processor} component (partition
         runtime) and its \texttt{process} component (partition address space)
         with the \texttt{Actual\_Processor\_Binding} property.
         
         \subsubsection{Scheduling}
         The scheduling policy of the partition is specified with the
         \texttt{POK::Scheduler} property in the \texttt{virtual processor}
         component (partition runtime).

         \subsubsection{Memory requirements}
         You can specify the memory requirements in two ways.

         First, with the \texttt{POK::Needed\_Memory\_Size} property on the
         process (partition address space). It will indicate the needed memory
         size for the process.

         You can also specify memory requirements with \aadl \texttt{memory}
         components. You bind a \texttt{memory} component to a partition process
         component with the \texttt{Actual\_Memory\_Binding} property. In that
         case, the properties (\texttt{Word\_Size}, \texttt{Word\_Count},
         \ldots) of the memory component will be used to generate its address
         space.

         \subsubsection{Additional features}
         You can specify which features are needed inside the partition (libc,
         libmath and so on). In that case, you have to specify them with the
         \texttt{POK::Additional\_Features} property.

      \subsection{Threads (ARINC653 processes)}
      Threads are contained in a partition. Thus, these components are contained
      in a \texttt{process} component (which models a partition). 

      There is the supported properties for threads declaration:
      \begin{itemize}
         \item
            \texttt{Source\_Stack\_Size}: the stack size of the thread
         \item
            \texttt{Period}: the actual period of the thread (execution rate)
         \item
            \texttt{Deadline}: the actual deadline of the thread (when the job
            should finish)
         \item
            \texttt{Compute\_Exeution\_Time}: the execution time needed to execute
            the application code of the threads.
      \end{itemize}

      \subsection{Inter-partitions channels}
         \subsubsection{Queuing ports}
         Queuing ports are mapped using \aadl event data ports connected between
         \aadl processes. This ports are also connected to \texttt{thread}
         components to send/receive data.

         \subsubsection{Sampling ports}
         Queuing ports are mapped using \aadl data ports connected between
         \aadl processes. This ports are also connected to \texttt{thread}
         components to send/receive data.

      \subsection{Intra-partitions channels}
         \subsubsection{Buffers}
         Buffers are mapped using \aadl event data ports connected between \aadl
         threads. This ports must not be connected outside the process.

         \subsubsection{Blackboards}
         Buffers are mapped using \aadl data ports connected between \aadl
         threads. This ports must not be connected outside the process.

         \subsubsection{Events}
         Buffers are mapped using \aadl event ports connected between \aadl
         threads. This ports must not be connected outside the process.

         \subsubsection{Semaphores}
         Semaphores are mapped using a shared \aadl data component between
         several \aadl \texttt{thread} components. The shared \texttt{data}
         component must use a concurrency protocol by defining the
         \texttt{Concurrency\_Control\_Protocol} property.

      \subsection{Protocols}
      You can describe which protocol you want to use in your system using a
      protocol layer. You specify the protocol layer using \texttt{virtual bus}
      components.

      FIXME -- complete once the work around virtual bus is finalized.

   \section{POK AADL library}
   POK provides an \aadl library for rapid prototyping of partitioned
   embedded architectures. This library contains predefines components
   associated with relevant properties to generate a partitioned architecture.

   The file that contains this \aadl library is located in
   \texttt{misc/aadl-library.aadl}.

   \section{Examples}
   Examples of \aadl models can be found in the \texttt{examples} directory of
   the \pok archive.
